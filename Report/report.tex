\documentclass[10pt]{report}

\begin{document}
q2)\\	
Simulating blockchain arrivals is vital for optimizing network performance, and the exponential distribution is a preferred model for several reasons. It inherently captures the random nature of block arrivals, mirroring the independence of mining processes. The memoryless property aligns with the assumption of independence between past and future events. This distribution's simplicity, with a single parameter (mean time between transactions), makes it easy to configure. Its flexibility allows control over transaction rates, exploring scenarios from low to high traffic. Additionally, empirical support suggests the exponential distribution is a reasonable approximation for blockchain applications, enhancing accurate system performance and scalability assessments.	
\\
q5)\\

The mean of dij is inversely proportional to cij in the context of the simulation to reflect the queuing delay at node i for forwarding to node j. As the link speed (cij) increases, the capacity for data transmission between nodes also rises. In turn, a higher link speed reduces the average time it takes for data to be transmitted, diminishing the queuing delay. Therefore, the inverse proportionality ensures that nodes with faster links experience lower queuing delays, aligning with the practical expectation that faster links lead to reduced congestion and more efficient data forwarding in the blockchain simulation. The use of exponential distribution for dij reflects the stochastic nature of queuing delays, capturing variability in the time it takes for nodes to forward data. This combination ensures a realistic portrayal of the dynamic and decentralized nature of blockchain network communication, aligning with real-world scenarios influenced by changing network conditions and demands.


\end{document}


































